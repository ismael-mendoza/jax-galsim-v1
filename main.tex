\documentclass[twocolumn,twocolappendix]{openjournal}
\renewcommand{\baselinestretch}{1.075}
\usepackage{macros} % define additional commands here
\shortauthors{Mendoza et al.}

%%%%%%%%%%%%%%%%%%%%%%%%%%%%%%%%%%%%%%%%%%%%%%%%
\begin{document}

\title{JAX-GalSim: A differentiable and gpu-accelerated approach to galaxy simulations}

\author{Ismael Mendoza \hskip2pt\href{https://orcid.org/0000-0002-6313-4597}{\includegraphics[width=9pt]{Orcid-ID.png}}}
\affiliation{Department of Physics, University of Michigan, Ann Arbor, MI 48105, USA}

\author{Jean-Eric Campagne \hskip2pt\href{https://orcid.org/0000-0002-1590-6927}{\includegraphics[width=9pt]{Orcid-ID.png}}}
\affiliation{Université Paris-Saclay, CNRS/IN2P3, IJCLab, 91405 Orsay, France}

\author{Benjamin Remy \hskip2pt\href{https://orcid.org/XXXX-XXXX-XXXX-XXXX}{\includegraphics[width=9pt]{Orcid-ID.png}}}
\affiliation{...}

\author{Matthew R. Becker \hskip2pt\href{https://orcid.org/0000-0001-7774-2246}{\includegraphics[width=9pt]{Orcid-ID.png}}}
\affiliation{High Energy Physics Division, Argonne National Laboratory, Lemont, IL 60439, USA}

\author{Francois Lanusse \hskip2pt\href{https://orcid.org/XXXX-XXXX-XXXX-XXXX}{\includegraphics[width=9pt]{Orcid-ID.png}}}
\affiliation{...}

\author{Axel Guinot \hskip2pt\href{https://orcid.org/0000-0002-5068-7918}{\includegraphics[width=9pt]{Orcid-ID.png}}}
\affiliation{Université Paris Cité, CNRS, AstroParticule et Cosmologie, F-75013, Paris, France}
\affiliation{McWilliams Center for Cosmology, Department of Physics, Carnegie Mellon University, Pittsburgh, PA 15213, USA}

\begin{abstract}
(...)
The \jgalsim code repository is public at \url{https://github.com/GalSim-developers/JAX-GalSim}.  
\end{abstract}

%%%%%%%%%%%%%%%%%%%%%%%%%%%%%%%%%%%%%%%%%
\section{Introduction} \label{sec:intro}

\begin{itemize}
    \item Modern Cosmological Surveys and how are Galaxy ``Simulations'' / ``Forward models'' useful (at a high-level): 
    
   
    
    \item Previous work on galaxy simulations, most prominently \galsim \citep{galsim2015}; what has been achieved and how they have been used.
    
    \item Previous work on differentiable or GPU galaxy simulations (AstroPhot? \cite{astrophot2023}, closest to \jgalsim probably), galaxy modeling with ML in general (BLISS, DebVader,...), introduce JAX \citep{jax2018github}, other initiatives in cosmology forward modeling (like jax-cosmo)
    
    \item motivation and introduction to \jgalsim 
    
    
In the near future, we will witness the production of large galaxy surveys that will usher Cosmology into a new era of high precision (e.g., DES, LSST, Euclid, etc.). Crucial questions will be addressed with increased acuity, such as the origin of the mechanism driving the acceleration of cosmic expansion, which the current cosmological standard model (aka LambdaCDM) attributes to a large fraction of Dark Energy component. More broadly, this will enhance our understanding of the various components of energy density. In this context, precise knowledge of cosmological parameters, such as those describing the equation of state of Dark Energy, will receive special attention. Current analyses involve comparing of observations to theory through reduced statistics, with the archetype being angular power spectra (aka $C_\ell$), as exemplified by the analyses of Cosmic Microwave Background data with likelihood function optimisation. However, this approach presents intrinsic difficulties, particularly in accounting for systematics. Although the unprecedented increase in data volumes to come, understanding and accounting such systematic effects will be more and more crucial. Consequently, analysis schemes increasingly emphasize Bayesian inferences from simulations (aka SBI). This forward modelling aims to produce nothing less than the final images delivered by telescopes, starting from cosmological parameters and accounting for all aspects such as the generation of large-scale structures (N-body simulation), galaxies, and the tracking of photons from these sources to the CCDs. In this framework, the introduction of various sources of systematic nuisances is more natural and shows principle advantages. Trials and implementations have already demonstrated the potential of such simulations. The challenge, of course, lies in the volume of data to be processed. 

A long this path, one has already identify that galaxy image imperfections can mimic the physical effect of weak lensing which in turn is a probe for gravitation property depending on cosmological theory parameters. \galsim  project\citep{galsim2015} was engaged to propose a common image simulation tool that can be shared across multiple surveys  aiming to get accurate estimation of galaxy properties from noisy images.

However, if we push the Simulation-Based Inference scheme further, as is already in use, the generation of Markov Chain Monte Carlo (aka MCMC) becomes a cornerstone that has been widely used and implemented in packages such as CosmoMC, CosmoSIS, MontePython, Cobaya, etc. But, in the scenario where simulation comes into play, the number of parameters tends to explode, leading to high-dimensional optimization. Thus, classical methods like Metropolis-Hastings will no longer be suitable, at least due to prohibitively long computation times. We will need to consider MCMC methods such as Hybrid Monte Carlo, where gradient calculations are involved. Currently, we are witnessing advancements in high-dimensional automatic differentiation (\textit{Autodiff}), as implemented in large machine learning models, that helps to compute safely the needed function derivatives.
Implementations of autodiff exist in widely used libraries such as TensorFlow, PyTorch, Stan, and more recently Julia and JAX. In this document, we present a partial implementation of GALSIM in JAX, simply called \jgalsim. 

The choice of JAX is motivated by several considerations. Although the API is not completely stabilized, it is noteworthy that JAX is the code language of the large Google language model (Gemini) trained on multi-TPU infrastructures, which ensures scalability for our application. By nature, JAX can differentiate functions written in Python (and Numpy), providing practical access due to the basic knowledge of the code development participants. This is a significant sociological aspect. But more conveniently, it also allows for recycling of some \galsim source code with little adaptation. We will address the differences in the text. JAX also provides mechanisms for vectorization (\texttt{vmap}) and parallelization, along with just-in-time compilation (\texttt{jit}), which optimize the code for specific hardware infrastructures like CPU, GPU, and TPU. JAX library contains in itself reimplementation of \texttt{Numpy} and \texttt{Scipy} methods (\texttt{jax.numpy} and \texttt{jax.scipy}) but also first it is one of the TensorFlow Probability backend from which we have used some special mathematical functions, and second it exists now a sizeable third party echosystem as for instance \texttt{JaxOpt} and \texttt{Optax} that gives access to hardware accelerated optimizers, and \texttt{Numpyro} a probabilistic programming language that is used in this paper.

    \item briefly describe and enumerate sections of the paper
\end{itemize}


%%%%%%%%%%%%%%%%%%%%%%%%%%%%%%%%%%%%%%%%
\section{Development Approach}
% Here we explain at a high-level how this package is different from GalSim and why it might be useful

%=====================================
\subsection{Differences with GalSim}

\begin{itemize}
    \item Differentiability 
    \item GPU
    \item JIT 
    \item Vectorizable
    \item Functional Oriented
    \item Pure python
    \item Randomness
\end{itemize}

%=====================================
\subsection{Potential Applications}

\begin{itemize}
    \item HMC 
    \item Cosmological Bayesian Pipelines
    \item NN-enabled realistic galaxy models
\end{itemize}

%=====================================
\subsection{Testing Suite}

Briefly describe our unit test suite. 

\begin{itemize}
    \item Comparisons with Galsim
    \item carefully checking every operation is jittable, differentiable, vectorizable
\end{itemize}

%%%%%%%%%%%%%%%%%%%%%%%%%%%%%%%%%
\section{Code Structure} % better name? 

\begin{itemize}
    \item What is currently implemented and important things missing
    
    \item Emphasize particularly tricky or interesting implementations of certain aspects of the code e.g. things that couldn't be replicated from GalSim due to JAX limitations
    
    \item Highlights: galaxy models implemented; image operations implemented such as shear, convolutions, interpolation; drawing techniques fft, real (?), pixel response 
\end{itemize}


%%%%%%%%%%%%%%%%%%%%%%%%%%%%%%%%%
\section{Benchmark with GalSim} 
% Accuracy 

In this section we compare the accuracy and speed of JAX-GalSim with GalSim \citep{galsim2015}, both in CPU and GPU, when rendering simple galaxy images.

\begin{figure}
\includegraphics[width=0.45\textwidth]{example-image-a}
\caption{
    (...)
    }
\label{fig:benchmark1}
\end{figure}


%%%%%%%%%%%%%%%%%%%%%%%%%%%%%%%%%
\section{Conclusions and Future Directions}

We have presented...


%%%%%%%%%%%%%%%%%%%%%%%%%%%%%%%%%%%%%%
\section*{Acknowledgements} 

IM acknowledges the support of the Special Interest Group on High Performance Computing (SIGHPC) Computational and Data Science Fellowship. IM acknowledges support from the Michigan Institute for Computational Discovery and Engineering (MICDE) Graduate Fellowship. IM acknowledges support from the Leinweber Center for Theoretical Physics Summer Fellowship.

We also acknowledge the use of \texttt{jax} \citep{jax2018github,frostig2018compiling}, \texttt{numpy} \citep{numpy2020}, \texttt{scipy} \citep{scipy2020}, \astropy \citep{astropy:2013,astropy:2018,astropy:2022}, \texttt{matplotlib} \citep{matplotlib2007}, and \galsim \citep{galsim2015}.


%%%%%%%%%%%%%%%%%%%%%%%%%%%%%%%%%%%%%%
\section*{Author Contributions} 

\begin{itemize}
    \item IM: 

    \item MB: 

    \item JEC: 

    \item FL: 

    \item BR: 

    \item AG: 
\end{itemize}

%%%%%%%%%%%%%%%%%%%%%%%%%%%%%%%%%%%
\appendix
\section*{First Appendix}\label{app:first}

%%%%%%%%%%%%%%%%%%%%%%%%%%%%%%%%%%%%%
\bibliography{references}{}
\bibliographystyle{aasjournal}


%%%%%%%%%%%%%%%%%%%%%%%%%%%%%%%%%%%%%
\end{document}
